\graphicspath{{./2-Arquitectura/capitulo3/}}

\Capitulo{Motivación}{

Este capítulo se centra en la capa de motivación de la arquitectura empresarial. Dicha capa aborda los elementos fundamentales que impulsan el cambio dentro de la organización, facilitando la comprensión de las aspiraciones, necesidades y restricciones que guían las decisiones estratégicas.

A lo largo del capítulo, se presentan los modelos propios de esta capa, examinando el papel que desempeñan en la alineación de los objetivos organizacionales con las actividades operativas. Desde los \textbf{stakeholders clave} y sus \textbf{drivers}, hasta los \textbf{principios organizativos} y los \textbf{objetivos estratégicos}, cada componente contribuye a una narrativa estructurada orientada al cambio.

Este análisis integral, basado en las buenas prácticas del ADM \cite{ADM_definition} y en la expresividad del lenguaje ArchiMate \cite{archimate}, busca ofrecer una \textbf{visión holística} de la motivación organizacional. A través de esta aproximación, se promueve una toma de decisiones estratégica y coherente, alineada con los valores fundamentales y con una visión a largo plazo.
}

%--------------Stakeholder-----------------------
\PuntoDeVista{Stakeholder}{imgs/Modelo 1.pdf}{
    Este punto de vista representa a los distintos interesados del sistema, cada uno con objetivos que reflejan conceptos de valor medibles y cuantificables. Esta representación permite verificar en cualquier momento si se avanza conforme a lo proyectado. \cite{definicion_capas}
}{imgs/Caso 1.pdf}{
    El caso ilustra el modelo de negocio propuesto, enfocado en maximizar la satisfacción del cliente mediante personalización, confort y una adecuada relación costo-beneficio.
}{0.7}{0.8}

%--------------Realización de Objetivos-----------------------
\PuntoDeVista{Realización de Objetivo}{imgs/Modelo 2.pdf}{
    Este modelo permite visualizar cómo se cumplen los objetivos mediante su definición en forma de requerimientos y restricciones. Asimismo, incluye el concepto de principio como propiedad del sistema motivada por los objetivos. \cite{definicion_capas}
}{imgs/Caso 2.pdf}{
    El caso expone un enfoque que garantiza una experiencia satisfactoria para el cliente, basada en la personalización de camisetas y considerando tanto aspectos creativos como logísticos.
}{0.7}{0.8}

%--------------Contribución de Objetivos-----------------------
\PuntoDeVista{Contribución de Objetivos}{imgs/Modelo 3.pdf}{
    Esta perspectiva incorpora objetivos, principios, requerimientos y restricciones, añadiendo el concepto de influencia, que permite establecer relaciones positivas o negativas entre dichos elementos. Esto facilita el análisis de impacto y la identificación de posibles conflictos entre objetivos. \cite{definicion_capas}
}{imgs/Caso 3.pdf}{
    El caso propone una experiencia que combina personalización del producto con una interacción eficiente en la plataforma, enfocándose en la facilidad de pago y uso.
}{0.8}{0.7}

%--------------Principios-----------------------
\PuntoDeVista{Principios}{imgs/Modelo 4.pdf}{
    Este modelo permite representar los principios organizacionales fundamentales junto con los objetivos que los motivan. Refleja los aspectos de mayor relevancia para los interesados y que deben preservarse durante el desarrollo del sistema. \cite{definicion_capas}
}{imgs/Caso 4.pdf}{
    El caso demuestra el compromiso de la organización con la comunicación clara y la entrega de productos originales y de calidad, sustentados por tecnología y materiales de vanguardia, con el objetivo de proporcionar una experiencia personalizada e innovadora.
}{0.3}{0.9}

%------------Realización de Requerimientos-------------------------
\PuntoDeVista{Realización de Requerimientos}{imgs/Modelo 5.pdf}{
    Esta perspectiva permite identificar cómo los objetivos se transforman en requerimientos detallados. Además, posibilita descubrir especializaciones que pueden haber pasado desapercibidas en etapas previas del modelado. \cite{definicion_capas}
}{imgs/Caso 5.pdf}{
    En el caso se detallan los requerimientos enfocados en los principios de calidad e innovación, que se consideran claves dentro de la propuesta de valor de la empresa.
}{0.7}{0.6}

%----------------------Motivación----------------------
\PuntoDeVista{Motivación}{imgs/Modelo 6.pdf}{
    Este modelo muestra los elementos motivacionales que fundamentan el desarrollo del sistema, los cuales surgen a partir de los objetivos de los stakeholders y se traducen en requerimientos concretos. \cite{archimate}
}{imgs/Caso 6.pdf}{
    El caso examina dos ejes centrales: creatividad y calidad. A partir de ellos se realiza un análisis estratégico (FODA) orientado a las necesidades y expectativas de los stakeholders principales, que incluyen tanto a compradores como a artistas.
}{0.6}{0.6}
