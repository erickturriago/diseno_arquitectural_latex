\graphicspath{{./2-Arquitectura/capitulo7/}}

\Capitulo{Tecnología}{

Esta capa describe la arquitectura tecnológica de la organización, abordando tanto la estructura como el comportamiento de sus componentes. El objetivo es representar cómo los elementos físicos y lógicos interactúan para soportar las capas superiores del modelo empresarial.

Para ello, se analizan los siguientes puntos de vista propuestos en ArchiMate \cite{definicion_capas}:
\begin{itemize}
    \item Tecnología
    \item Uso de la Tecnología
    \item Despliegue e Implementación
    \item Estructura de Información
    \item Realización del Servicio
    \item Físico
    \item Capas
\end{itemize}
}

%--------------Tecnología-----------------------
\PuntoDeVista{Punto de Vista de Tecnología}{imgs/Modelo 1.pdf}{
    Este punto de vista incluye los elementos de software y hardware que respaldan la capa de aplicación, como dispositivos físicos, redes, sistemas operativos, bases de datos o middleware. \cite{definicion_puntos_de_vista}
}{imgs/Caso 1.pdf}{
    El caso muestra la infraestructura tecnológica que sustenta las operaciones de “Estampa tu Idea”. Se representan componentes como servidores, bases de datos PostgreSQL, sistema operativo Windows y redes LAN e Internet, evidenciando cómo se articulan para soportar las funcionalidades del negocio y la aplicación.
}{0.8}{0.8}

%--------------Uso de Tecnología-----------------------
\PuntoDeVistaResponsive{Punto de Vista de Uso de Tecnología}{imgs/Modelo 2.pdf}{
    Este modelo representa cómo las aplicaciones son soportadas por la infraestructura tecnológica. Incluye los servicios provistos por dispositivos, software y redes, y permite evaluar su rendimiento frente a las demandas de las aplicaciones que utilizan dichos servicios. \cite{definicion_puntos_de_vista}
}{imgs/Caso 2.pdf}{
    El caso ilustra la estructura tecnológica que respalda procesos como pagos e inventario. Se incluyen elementos como la pasarela de pagos, el sistema de software del servidor y plugins asociados. La representación muestra cómo se integran estos componentes para ofrecer funcionalidades específicas dentro de la plataforma.
}{0.6}{0.6}

%--------------Despliegue e Implementación-----------------------
\PuntoDeVistaResponsive{Punto de Vista de Despliegue e Implementación}{imgs/Modelo 2.pdf}{
    Este punto de vista muestra cómo las aplicaciones se implementan sobre la infraestructura tecnológica. Describe la relación entre componentes lógicos y artefactos físicos, permitiendo analizar aspectos como escalabilidad o rendimiento. \cite{definicion_puntos_de_vista}
}{imgs/Caso 3.pdf}{
    El caso aborda la gestión de bases de datos e inventarios, incluyendo nodos, artefactos y redes que representan los dispositivos, datos y conexiones necesarios. Se destaca cómo la infraestructura tecnológica respalda el funcionamiento coordinado de los sistemas que soportan las operaciones internas.
}{0.6}{0.5}

%--------------Estructura de Información-----------------------
\PuntoDeVista{Punto de Vista de Estructura de Información}{imgs/Modelo 4.pdf}{
    Este modelo permite visualizar cómo la información es estructurada y respaldada a nivel tecnológico. Resulta útil para el análisis de rendimiento y para comprender la relación entre datos y componentes de software o hardware. \cite{definicion_capas}
}{imgs/Caso 4.pdf}{
    El caso muestra la estructura de información del sistema, conectando componentes como “Estampado”, “Catálogo”, “Inventario” y “Ubicación”. Se evidencian relaciones entre procesos y datos, como la gestión de materiales y la localización de clientes, que son esenciales para operar la plataforma.
}{0.7}{0.7}

%--------------Realización del Servicio-----------------------
\PuntoDeVistaResponsive{Punto de Vista de Realización del Servicio}{imgs/Modelo 5.pdf}{
    Este modelo describe cómo la infraestructura tecnológica permite la prestación de servicios requeridos por las aplicaciones y procesos de negocio. Facilita el diseño de soluciones que respondan a necesidades funcionales y operativas. \cite{definicion_capas}
}{imgs/Caso 5.pdf}{
    El caso presenta el flujo tecnológico que permite la personalización de camisetas. Desde la configuración inicial por parte del cliente hasta la creación del diseño por el artista, se evidencian los elementos que intervienen: catálogo, inventario, preferencias de usuario y recursos técnicos. El modelo refleja una ejecución integrada y eficiente.
}{0.7}{0.7}

%--------------Físico-----------------------
\PuntoDeVista{Punto de Vista Físico}{imgs/Modelo 6.pdf}{
    Este modelo se enfoca en representar los elementos tangibles que conforman la arquitectura tecnológica de la organización, incluyendo dispositivos, servidores y otros recursos físicos necesarios para soportar las operaciones.
}{imgs/Caso 6.pdf}{
    El caso muestra la interacción entre el servidor web, el despachador y los proveedores. El despachador distribuye materias primas hacia el inventario, mientras que el servidor garantiza el funcionamiento de la plataforma. Esta vista permite comprender cómo los recursos físicos participan activamente en los servicios entregados.
}{0.8}{0.8}

%--------------Capas-----------------------
\PuntoDeVista{Punto de Vista de Capas}{imgs/Modelo 7.pdf}{
    Este diagrama presenta una visión integrada de la arquitectura, organizando los elementos en capas: negocio, aplicación y tecnología. La separación entre capas y servicios permite analizar cómo se conectan los distintos niveles para entregar valor. \cite{definicion_capas}
}{imgs/Caso 7.pdf}{
    El caso refleja cómo las capas interactúan para brindar el servicio de personalización y venta de camisetas. Desde la infraestructura tecnológica hasta los servicios del negocio, se evidencia una continuidad que permite al usuario acceder a la plataforma, personalizar su prenda y concretar la compra.
}{0.3}{0.5}
