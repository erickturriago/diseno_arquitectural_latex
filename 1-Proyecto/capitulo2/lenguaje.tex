\graphicspath{{./1-Proyecto/capitulo2/}}

\chapter{Archimate}{
    \section{Introducción}
    En el contexto de la Arquitectura Empresarial, contar con un lenguaje de modelado estandarizado resulta esencial para representar de manera clara y coherente los diferentes dominios organizacionales. En este sentido, Archimate surge como una solución práctica y ampliamente adoptada que permite modelar, describir, analizar y comunicar arquitecturas complejas de forma estructurada.

    ArchiMate es un lenguaje de modelado abierto, desarrollado y respaldado por The Open Group, que proporciona una notación visual unificada para representar las relaciones entre procesos de negocio, aplicaciones y tecnología. Su objetivo es facilitar la comprensión de la arquitectura empresarial tanto para los equipos técnicos como para los diferentes grupos de interés de una organización, permitiendo una visión integral y coherente del estado actual y futuro de la empresa.

    Gracias a su enfoque estructurado, Archimate permite visualizar cómo los cambios en un área afectan a otras, promoviendo la alineación estratégica entre el negocio y las tecnologías de la información. Asimismo, ofrece una base sólida para el análisis de impacto, la toma de decisiones y la gestión de transformaciones organizacionales.

    Este capítulo explora las bases conceptuales de Archimate, sus principales características, y su papel como lenguaje complementario dentro del desarrollo de la arquitectura empresarial con TOGAF.
	\section{¿Qué es Archimate?}
    \textbf{ArchiMate es un lenguaje de modelado de arquitectura empresarial} abierto e independiente para respaldar la descripción, el análisis y la visualización de la arquitectura dentro y entre dominios comerciales de manera inequívoca.\\\
    ArchiMate ofrece un lenguaje común para describir la construcción y operación de procesos comerciales, estructuras organizativas, flujos de información, sistemas de TI e infraestructura técnica. Esta información ayuda a las partes interesadas a diseñar, evaluar y comunicar las consecuencias de las decisiones y los cambios dentro y entre estos dominios comerciales. \cite{Archimate_definition}
	\section{Características}	
	ArchiMate es un  estándar abierto  mantenido y actualizado por The Open Group. Se tienen en cuenta los últimos desarrollos e ideas en arquitectura empresarial y el marco ArchiMate se mejora continuamente. Algunas de las características que posee Archimate son:
    \begin{itemize}
        \item ArchiMate garantiza la coherencia en todos los modelos de arquitectura, por lo que es un lenguaje ágil y sencillo. 
        \item Contiene suficientes conceptos para modelar la arquitectura empresarial y no incluye todos los conceptos posibles para no salirse de sus propios límites. Como resultado, la arquitectura empresarial se puede comunicar de manera clara y coherente en todos los dominios de su negocio. 
        \item Su estructura uniforme hace que sea fácil de aprender y aplicar.
        \item ArchiMate permite realizar un modelado de alto nivel dentro de un dominio, es también bases para el análisis de identificación de procesos, actores, entre otros elementos involucrados en una arquitectura empresarial, este lenguaje se ofrece así como un complemento que ofrece metodologías que permiten desarrollar una arquitectura empresarial.
        \item ArchiMate ofrece una forma de generalización de comunicación a nivel empresarial, lo que potencializa la velocidad con la cual se puede conocer un proceso o elemento que pertenece a una arquitectura empresarial.
    \end{itemize}
}