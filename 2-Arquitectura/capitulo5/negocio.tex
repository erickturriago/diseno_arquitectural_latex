\graphicspath{{./2-Arquitectura/capitulo5/}}

\Capitulo{Negocio}{

En el entorno empresarial, la Capa de Negocio representa el espacio donde las estrategias se concretan en acciones y operaciones visibles. Este capítulo examina su estructura, enfocándose en cómo los procesos, la organización interna y la propuesta de valor se articulan de forma coherente.

A lo largo del documento se analizan distintos modelos, desde la identificación de actores y roles hasta la descripción detallada de procesos y servicios. Cada componente contribuye a establecer una relación clara entre lo estratégico y lo operativo, asegurando que la ejecución mantenga coherencia con los propósitos institucionales.

Más que reflejar una estrategia, esta capa participa activamente en su construcción diaria. La forma en que se ofrecen productos, se gestiona el talento y se configuran los flujos de trabajo permite comprender cómo se crea y se entrega valor.

El enfoque adoptado se basa en las prácticas definidas por el ADM \cite{ADM_definition} y busca aportar una lectura clara del funcionamiento interno de la organización, facilitando el análisis, la mejora continua y el diseño de soluciones sostenibles para los desafíos del negocio.
}

%--------------Organización -----------------------
\newpage
\PuntoDeVista{Organización}{imgs/Modelo 1.pdf}{
    Este modelo describe la estructura organizativa de la empresa, junto con las relaciones entre roles, actores y colaboraciones. Su propósito es representar cómo se distribuyen las responsabilidades y cómo se conectan los elementos humanos dentro de la arquitectura de negocio. \cite{definicion_capas}
}{imgs/Caso 1.pdf}{
    El caso muestra la interacción entre artistas y clientes mediante la plataforma “Estampa tu idea”. Los artistas suben sus diseños, y los clientes pueden explorarlos y adquirirlos. Este modelo fomenta la personalización y abre una vía directa para que los creadores ofrezcan sus productos, alineándose con principios como creatividad, transparencia e innovación.
}{0.7}{0.55}

%--------------Cooperación de Actor-----------------------
\PuntoDeVista{Cooperación de Actor}{imgs/Modelo 2.pdf}{
    Este modelo permite visualizar cómo los distintos actores interactúan para llevar a cabo actividades de negocio. Al representar la cooperación entre roles, facilita la comprensión del funcionamiento interno y de las dinámicas de colaboración en la organización. \cite{archimate}
}{imgs/Caso 2.pdf}{
    El caso representa la cooperación entre los distintos actores del modelo “Estampa tu idea”, vinculando la creatividad de los artistas con las preferencias de los clientes. Se destacan valores como la innovación, la calidad del servicio y el compromiso con la experiencia del usuario.
}{0.7}{0.8}

%--------------Función de Negocio-----------------------
\PuntoDeVista{Función de Negocio}{imgs/Modelo 3.pdf}{
    Este modelo muestra cómo se asignan responsabilidades a los distintos roles dentro de la organización. Las flechas indican el flujo desde el actor hasta la función, reflejando la ejecución de tareas asociadas a cada rol. \cite{definicion_capas}
}{imgs/Caso 3.pdf}{
    El caso evidencia que las funciones de negocio están interrelacionadas y se orientan hacia un objetivo común: ofrecer una experiencia personalizada de alta calidad. La participación activa de los diseñadores, que asumen el rol de artistas, refuerza el enfoque creativo de la organización.
}{1}{1}

%------------Proceso de Negocio-------------------------
\PuntoDeVista{Proceso de Negocio}{imgs/Modelo 4-5.pdf}{
    Este modelo permite representar el conjunto de actividades que conforman un proceso de negocio. Su enfoque facilita identificar dependencias, actores involucrados y posibles puntos de mejora. \cite{archimate}
}{imgs/Caso 4.png}{
    El caso aborda el proceso de venta de camisetas, desde la recepción del pedido hasta su entrega. Incluye validaciones internas y etapas logísticas que garantizan una operación ordenada y orientada a cumplir con los tiempos y requerimientos del cliente.
}{0.8}{0.9}

%----------------Cooperación de Proceso de Negocio----------------------
\PuntoDeVista{Cooperación de Proceso de Negocio}{imgs/Modelo 4-5.pdf}{
    Este modelo se centra en las interacciones que se producen durante la ejecución de procesos compartidos entre actores. Permite entender cómo distintas áreas colaboran para lograr un resultado común. \cite{definicion_capas}
}{imgs/Caso 5.pdf}{
    El caso describe el proceso desde la solicitud realizada por el cliente hasta la entrega del producto, involucrando a proveedores y artistas en una cadena coordinada. Se destaca la articulación entre creatividad y logística como elementos que garantizan la respuesta oportuna.
}{0.8}{0.9}

%----------------Producto----------------------
\PuntoDeVista{Producto}{imgs/Modelo 6.pdf}{
    Este modelo proporciona una visión completa de los elementos que rodean un producto, incluyendo contratos, eventos, servicios y procesos asociados. Sirve para trazar cómo se construye y entrega la oferta empresarial. \cite{archimate}
}{imgs/Caso 6.pdf}{
    El caso aborda el producto “Venta de camisetas con diseños personalizados” desde distintas perspectivas: la plataforma digital, los elementos diferenciadores y los servicios asociados. Se incluyen mecanismos que mejoran la experiencia del cliente, como recomendaciones de estampados y opciones de pago flexibles. Esta visión integrada resulta útil para planificar y rediseñar tanto servicios como procesos en función del producto ofrecido.
}{0.8}{0.6}
