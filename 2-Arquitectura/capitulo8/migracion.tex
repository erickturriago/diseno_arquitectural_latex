\graphicspath{{./2-Arquitectura/capitulo8/}}

\Capitulo{Migración e implementación}{

Este capítulo aborda la gestión del cambio en la arquitectura empresarial, específicamente el proceso de transición desde un estado actual hacia un estado deseado. Dicho proceso, conocido como migración, tiene implicaciones tanto a corto como a largo plazo, y afecta la evolución estratégica de la organización, así como la toma de decisiones durante su implementación. \cite{farmaweb_case}
}

%--------------Punto de Vista de Proyecto-----------------------
\newpage
\PuntoDeVista{Punto de Vista de Proyecto}{imgs/Modelo 1.pdf}{
    Este modelo permite representar la gestión del cambio arquitectónico. La migración entre arquitecturas suele requerir una planificación gradual, considerando que los sistemas existentes deben seguir operativos. También deben contemplarse aspectos como el cumplimiento de estándares, la adaptación del personal, y las implicaciones legales y financieras. \cite{definicion_capas}
}{imgs/Caso 1.pdf}{
    El caso representa la forma en que las decisiones estratégicas planteadas en las capas anteriores se consolidan a través del desarrollo del portal web “EstampaTuIdea.com”, proyectando su implementación como eje central del cambio arquitectónico.
}{1}{1}

%--------------Punto de Vista de Migración-----------------------
\PuntoDeVista{Punto de Vista de Migración}{imgs/Modelo 2.pdf}{
    Este punto de vista permite modelar el paso progresivo entre una arquitectura actual y una futura. Se utilizan conceptos como mesetas, que representan estados temporales de la arquitectura, y brechas, que identifican los elementos a resolver entre dichos estados. \cite{definicion_capas}
}{imgs/Caso 2.pdf}{
    El caso presenta una versión preliminar del portal web “EstampaTuIdea.com” y resalta los elementos estructurales que deben ser tenidos en cuenta para su construcción. Se plantea como una etapa de transición en el proceso de evolución arquitectónica.
}{0.6}{0.8}

%--------------Punto de Vista de Implementación/Migración-----------------------
\PuntoDeVista{Punto de Vista de Implementación/Migración}{imgs/Modelo 3.pdf}{
    Este modelo vincula programas y proyectos con los componentes de la arquitectura que buscan materializar. Su enfoque facilita la administración del portafolio de iniciativas, conectando requerimientos de negocio con acciones específicas de implementación. \cite{definicion_capas}
}{imgs/Caso 3.pdf}{
    El caso sintetiza lo expuesto en los modelos anteriores, mostrando cómo la estructura planteada en el punto de vista de migración sirve de base para alcanzar los objetivos estratégicos del proyecto.
}{0.7}{0.7}
