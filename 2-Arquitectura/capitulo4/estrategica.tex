\graphicspath{{./2-Arquitectura/capitulo4/}}

\Capitulo{Estrategia}{

En el panorama empresarial contemporáneo, la definición y ejecución de estrategias se consolida como un pilar esencial para alcanzar el éxito organizacional. Dentro del marco de la arquitectura empresarial, la capa de estrategia proporciona el entorno adecuado para formular y alinear los objetivos estratégicos, ofreciendo una base sólida para la toma de decisiones orientadas al crecimiento sostenible.

Este capítulo profundiza en el análisis de la capa de estrategia, aprovechando el potencial expresivo del lenguaje ArchiMate \cite{archimate}. \textbf{Esta capa actúa como un punto de convergencia donde se articulan metas de alto nivel, principios rectores e iniciativas estratégicas, definiendo así el rumbo de la organización.}

A lo largo del documento se examinan en detalle los modelos incluidos en esta capa. Desde la identificación de los \textbf{stakeholders estratégicos} hasta la definición de \textbf{principios clave} y la especificación de \textbf{metas estratégicas}, cada modelo desempeña un papel dentro de la consolidación de la visión organizacional.

El propósito de este análisis es ofrecer una visión estructurada e integral de la estrategia empresarial, utilizando elementos gráficos y descriptivos para representar la complejidad inherente al proceso estratégico. Basado en las buenas prácticas del ADM \cite{ADM_definition}, este enfoque busca facilitar la alineación entre estrategia y ejecución, fortaleciendo la capacidad de la organización para alcanzar ventajas competitivas sostenibles.

Por medio de esta capa, se evidencia cómo la arquitectura empresarial puede funcionar como un faro orientador para la toma de decisiones estratégicas, especialmente en contextos dinámicos y altamente competitivos.
}

%--------------Mapa de Capacidad-----------------------
\PuntoDeVista{Mapa de Capacidad}{imgs/Modelo 1.pdf}{
    Este punto de vista representa una estructura conceptual que demuestra cómo los recursos y las capacidades se integran sinérgicamente para alcanzar resultados estratégicos dentro de la organización. \cite{definicion_capas}
}{imgs/Caso 1.pdf}{
    El caso ilustra cómo la gestión de tecnologías de la información, la administración de recursos humanos y la implementación de herramientas tecnológicas se articulan estratégicamente para elevar la satisfacción con el producto final. Este enfoque destaca la importancia de coordinar de manera eficiente los recursos clave, garantizando así que la entrega del producto cumpla con los estándares requeridos y supere las expectativas del usuario.
}{0.5}{0.7}

%--------------Realización de Resultados-----------------------
\PuntoDeVista{Realización de Resultados}{imgs/Modelo 2.pdf}{
    Este modelo representa la estructura y relaciones internas de la organización, permitiendo visualizar cómo se integran roles, actores y colaboraciones. Su finalidad es clarificar cómo estos elementos sostienen la arquitectura de negocio. \cite{definicion_capas}
}{imgs/Caso 2.png}{
    El caso evidencia la correlación entre la calidad de los productos y servicios ofrecidos y el nivel de satisfacción del cliente. Esta relación se ve reforzada por una gestión eficiente y una operación optimizada de las tecnologías de la información.
}{0.6}{0.6}

%--------------Mapa de Recurso-----------------------
\PuntoDeVista{Mapa de Recurso}{imgs/Modelo 3.pdf}{
    Este modelo muestra la relación entre recursos y capacidades dentro del marco estratégico, constituyendo una herramienta útil para la planificación y gestión de los medios necesarios para alcanzar los objetivos de la organización. \cite{definicion_capas}
}{imgs/Caso 3.pdf}{
    El caso representa la integración efectiva de recursos humanos y tecnológicos en la construcción de una tienda virtual. Se resalta la coordinación entre personas y tecnologías, lo cual permite maximizar el uso de los recursos disponibles y alcanzar con éxito los objetivos del proyecto.
}{0.6}{0.7}

%------------Flujo de Valor-------------------------
\PuntoDeVista{Flujo de Valor}{imgs/Modelo 4.png}{
    Esta perspectiva permite analizar cómo los recursos y actividades se interrelacionan para generar valor dentro de la estrategia organizacional. Facilita una visión integral para la planificación y mejora de procesos, con el objetivo de maximizar el valor ofrecido por la organización. \cite{archimate}
}{imgs/Caso 4.pdf}{
    El flujo de valor comienza con la creatividad del artista, transformando ideas en diseños de alto potencial. Esta creatividad se conecta con las necesidades del mercado, donde la personalización impulsa vínculos emocionales con el cliente. Esta interacción fortalece la competitividad del producto, culminando en altos niveles de satisfacción, fidelización y reconocimiento de marca.
}{0.4}{0.8}

%----------------------Estrategia----------------------
\PuntoDeVistaResponsive{Estrategia}{imgs/Modelo 5.pdf}{
    El modelo de estrategia permite alinear iniciativas estratégicas con recursos y capacidades organizacionales, asegurando que las decisiones estén fundamentadas en una visión coherente y orientada al logro de objetivos. \cite{ADM_definition}
}{imgs/Caso 5.pdf}{
    Este caso presenta una integración entre políticas empresariales sólidas, estrategia de marketing, recursos humanos, tecnologías de información y recursos económicos. Todo ello sustentado en una plataforma denominada “Estampa tu idea”, donde los usuarios pueden personalizar sus camisetas y los artistas colaborar creativamente. Esta plataforma se convierte en el eje central de la propuesta de valor, generando conexión con los clientes y diferenciación en el mercado.
}{0.6}{0.8}
