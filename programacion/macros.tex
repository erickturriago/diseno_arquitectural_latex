% Contenido para programacion/macros.tex

\newcommand{\Capitulo}[2]{
\chapter{#1}
#2
}

\lstset{ %
  keywordstyle=\color{blue},  
}


\newcommand{\PuntoDeVista}[7]{
\section{#1}
\subsection{Modelo}
\begin{figure}[H]
	\centering	
    \includegraphics[width={#6}\linewidth]{#2}
	\caption{Modelo #1}
\end{figure}
#3
\subsection{Caso}
\begin{figure}[H]
	\centering	
    \includegraphics[width={#7}\linewidth]{#4}
	\caption{Caso #1}	
 \end{figure}
 #5 
}



\newcommand{\PuntoDeVistaResponsive}[7]{
\begin{center}
  \begin{minipage}{1\textwidth}
\section{#1}

\subsection{Modelo}

\begin{figure}[H]
	\centering	
    \includegraphics[width=#6\linewidth]{#2}
	\caption{Modelo #1}
\end{figure}
#3
\end{minipage}
\end{center}
\begin{center}
  \begin{minipage}{1\textwidth}
\subsection{Caso}
\begin{figure}[H]
	\centering	
    \includegraphics[width=#7\linewidth]{#4}
	\caption{Caso #1}	
 \end{figure}
#5
\end{minipage}
\end{center}
}

\newcommand{\Patron}[9]{%
    \section{#1}
    \subsection{Realización}
    #2
    \subsection{Funcionamiento}
    #3
    \subsection{Estructura}
    \subsubsection{Modelo}
    \begin{figure}[H]
        \centering
        \includegraphics[width=#7\linewidth]{#4}
        \caption{Modelo del patrón #1.}
    \end{figure}
    \subsubsection{Aplicación}
    \begin{figure}[H]
        \centering
        \includegraphics[width=#8\linewidth]{#5}
        \caption{Aplicación del patrón #1.}
    \end{figure}
    \subsection{Código}
    \inputminted{#9}{#6} % #6 es el archivo fuente
}