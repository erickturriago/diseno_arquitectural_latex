\graphicspath{{./2-Arquitectura/capitulo7/}}

\Capitulo{Tecnología}{
	
En esta capa podemos observar la arquitectura tecnológica de la organización, y se describe la estructura y el comportamiento de cada uno de sus elementos.
Para poder observar esto se proponen los siguientes puntos de vista.\cite{definicion_capas}:
\begin{itemize}
    \item Punto de Vista de Tecnología
    \item Punto de Vista de Uso de la Tecnología
    \item Punto de Vista de Despliegue e implementación
    \item Punto de Vista de Estructura de información
    \item Punto de Vista de Realización del servicio
    \item Punto de Vista Físico
    \item Punto de Vista de Capas

\end{itemize}
}

%--------------Punto de Vista de Tecnología-----------------------
\PuntoDeVista{Punto de Vista de Tecnología}{imgs/Modelo 1.pdf}
{
   El punto de vista de tecnología contiene los elementos de software y hardware que soportan la capa de aplicación, como dispositivos físicos, redes o sistemas de software. (sistemas operativos, bases de datos o middleware).
\cite{definicion_puntos_de_vista}
}
{imgs/Caso 1.pdf}
{
    Este caso representa la infraestructura tecnológica necesaria para respaldar las operaciones de Estampa tu Idea. Los componentes específicos, como los servidores, la base de datos PostgreSQL, el sistema operativo Windows y las conexiones a Internet y LAN, \textbf{interactúan entre sí para formar una infraestructura cohesiva que respalda las capas superiores }de negocio y aplicación.
 }{1.1}{1}

%--------------Punto de Vista de Uso de Tecnología-----------------------
\PuntoDeVistaResponsive{Punto de Vista de Uso de Tecnología}{imgs/Modelo 2.pdf}{
   El punto de vista de uso de tecnología muestra como las aplicaciones son soportadas por la infraestructura de software y hardware: los servicios de infraestructura son entregados por los dispositivos, los sistemas de software y redes son entregados a las aplicaciones. Este punto de vista juega un rol importante en el análisis del rendimiento y la escalabilidad y puede ser usado para determinar requerimientos de rendimiento y calidad en la infraestructura basado en las demandas de las aplicaciones que la usan. \cite{definicion_puntos_de_vista}
}{imgs/Caso 2.pdf}{
   \textbf{ Este caso representa el flujo tecnológico para procesos de pagos e inventario.} La aplicación del servidor es un sistema de software, la pasarela de pagos es un servicio de infraestructura y el plugin de pagos es un componente de software. Esta representación detalla la integración de componentes clave para funciones específicas, \textbf{esencial para una eficaz gestión tecnológica en la organización.}
}{1.1}{1}

%--------------Punto de Vista de Despliegue e Implementacion-----------------------
\PuntoDeVistaResponsive{Punto de Vista de Despliegue e Implementación}{imgs/Modelo 2.pdf}{
  El punto de vista de implementación y despliegue muestra como uno o más aplicaciones son realizadas sobre la infraestructura. Esto implica el mapeo de aplicaciones (lógicas) y componentes en artefactos (fisicos). Esta vista juega un papel importante en el análisis del rendimiento y la escalabilidad debido a la relación entre la infraestructura y el mundo lógico de las aplicaciones. \cite{definicion_puntos_de_vista}
}{imgs/Caso 3.pdf}{
    Este caso representa la gestión de bases de datos e inventarios, destacando la interacción entre diferentes componentes. Abarca nodos de infraestructura, artefactos y redes, que reflejan dispositivos físicos, datos y conectividad respectivamente.\textbf{ Este enfoque ofrece una visión clara de cómo los sistemas de software se integran con la infraestructura de hardware, permitiendo una gestión efectiva de recursos para respaldar las operaciones organizacionales.}
}{1}{1}




%---------------------- Punto de Vista de Estructura de Informaciónn----------------------
\PuntoDeVista{ Punto de Vista de Estructura de Información}{imgs/Modelo 4.pdf}
{
    En este punto de vista se observa como la aplicación es soportada tanto a nivel de software como de hardware. Gracias a esto se podrá usar este punto de vista para el análisis de rendimiento y escalabilidad.
\cite{definicion_capas}

}
{imgs/Caso 4.pdf}
{
   Este caso representa la estructura de la información que se despliega mediante componentes interconectados. Por ejemplo, Estampado se vincula con Catálogo, indicando la disponibilidad de productos. Materiales conecta directamente con Inventario y Cliente, gestionando recursos y relaciones con los clientes. La Ubicación interactúa con los clientes, señalando su geolocalización. La Fábrica se relaciona con el Inventario, gestionando la producción. Esta estructura revela cómo los componentes tecnológicos se relacionan para respaldar las operaciones del sistema.
    
}{0.7}{0.7}


%----------------------  Punto de Vista de Realización del Servicio----------------------
\PuntoDeVistaResponsive{ Punto de Vista de Realización del Servicio}{imgs/Modelo 5.pdf}
{
 En este punto de vista se observa como la aplicación soporta varios procesos de negocio y como es requerida por otras aplicaciones. Gracias a este punto de vista se puede
observar las necesidades de los diferentes procesos de negocios en cuanto a los servicios que ofrecen sus aplicaciones y esto facilita tanto un mejor diseño de las aplicaciones como
el entendimiento mismo de procesos.\cite{definicion_capas}

}
{imgs/Caso 5.pdf}
{
   \textbf{Este caso representa como la realización del servicio de  personalización se despliega con un flujo eficiente.} Desde la Experiencia de Personalización Avanzada hasta la Creación de estampados únicos por parte del Artista, cada etapa se conecta de manera fluida. El cliente inicia el proceso, configura sus preferencias, y el catálogo y el inventario facilitan la selección de opciones. Finalmente, el artista utiliza los recursos del inventario para concretar la solicitud del cliente, asegurando una experiencia personalizada y satisfactoria. 
}{1.2}{0.8}


%----------------------Punto de vista físico---------------------
\PuntoDeVista{Punto de vista físico
}{imgs/Modelo 6.pdf}
{
   Este punto de vista se concentra en representar los elementos tangibles de la arquitectura empresarial. Este modelo describe la infraestructura física y tecnológica necesaria para ejecutar los procesos y servicios empresariales.

}
{imgs/Caso 6.pdf}
{
   El caso muestra cómo los componentes físicos (servidor web, despachador y proveedores) interactúan para proporcionar servicios a los usuarios. El despachador juega un papel crucial al distribuir las materias primas entrantes a los inventarios del servidor web.
}{0.8}{0.8}

%----------------------Punto de Vista de Capas----------------------
\PuntoDeVista{Punto de Vista de Capas
}{imgs/Modelo 7.pdf}
{
  En este diagrama se pueden describir varios aspectos de la arquitectura empresarial.
Se pueden categorizar estas capas en dos tipos: las dedicadas y las de servicio. Esta división resulta de agrupar las diferentes relaciones y el cómo expone sus servicios a la siguiente capa. El punto principal de esta capa es dar una vista general en un solo diagrama.\cite{definicion_capas}

}
{imgs/Caso 7.pdf}
{
   Este caso representa como las capas de Tecnología, Aplicación y Negocio interactúan las unas con las otras para poder proveer un servicio al negocio: El cual es dar la posibilidad de comprar y personalizar camisetas.
    
}{0.5}{0.5}
