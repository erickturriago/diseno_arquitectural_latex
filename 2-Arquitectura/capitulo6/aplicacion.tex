\graphicspath{{./2-Arquitectura/capitulo6/}}

\Capitulo{Aplicación}{

La Capa de Aplicación permite modelar el sistema de información de la organización, incluyendo cómo las aplicaciones se relacionan entre sí para respaldar las actividades del negocio. Este enfoque facilita la comprensión del comportamiento, cooperación, estructura y uso de las aplicaciones dentro del entorno organizacional.

A continuación, se abordan los siguientes puntos de vista propuestos por el marco ArchiMate, orientados al análisis técnico de las aplicaciones \cite{definicion_capas}:
\begin{itemize}
    \item Comportamiento de la Aplicación
    \item Cooperación de Aplicación
    \item Estructura de Aplicación
    \item Uso de Aplicación
\end{itemize}
}

%--------------Comportamiento de la Aplicación-----------------------
\PuntoDeVistaResponsive{Comportamiento de la Aplicación}{imgs/Modelo 1.pdf}{
    Este punto de vista permite representar el comportamiento interno de una aplicación al ejecutar uno o varios servicios. Resulta útil para diseñar funcionalidades principales o para identificar posibles solapamientos entre componentes. Cada componente encapsula una parte específica del sistema, manteniendo una estructura modular y reemplazable. \cite{definicion_puntos_de_vista}
}{imgs/Caso 1.pdf}{
    El caso muestra el comportamiento de la aplicación encargado de gestionar el proceso de personalización de camisetas. Incluye la publicación de diseños por parte de los artistas, la selección y modificación de modelos por los clientes, y la posterior compra. Este flujo refleja los elementos esenciales de un sistema de comercio electrónico.
}{0.85}{0.85}

%--------------Cooperación de Aplicación-----------------------
\PuntoDeVista{Cooperación de Aplicación}{imgs/Modelo 2.pdf}{
    Este modelo se enfoca en las relaciones entre los componentes del sistema, ya sea mediante flujos de información o servicios utilizados u ofrecidos. Suele emplearse para visualizar la cooperación general entre aplicaciones o la orquestación de servicios que respaldan un proceso empresarial. \cite{definicion_puntos_de_vista}
}{imgs/Caso 2.pdf}{
    En este caso se ilustra la colaboración entre componentes en dos áreas: interacción con usuarios (Front Office) y gestión interna (Back Office). Cada módulo tiene una función específica dentro del conjunto, contribuyendo a una operación coordinada de la plataforma.
}{0.6}{0.6}

%--------------Estructura de Aplicación-----------------------
\PuntoDeVista{Estructura de Aplicación}{imgs/Modelo 3.pdf}{
    Este modelo describe la estructura de una o varias aplicaciones y sus componentes. Facilita la comprensión del diseño modular, así como de las interfaces y objetos que interactúan entre ellos. \cite{definicion_capas}
}{imgs/Caso 3.pdf}{
    El caso presenta las interfaces a través de las cuales la página web interactúa con los distintos componentes del sistema. Se expone la dinámica de comunicación interna de la aplicación, mostrando cómo se organiza el flujo de información entre módulos.
}{0.8}{0.9}

%----------------------Uso de Aplicación----------------------
\PuntoDeVista{Uso de Aplicación}{imgs/Modelo 4.pdf}{
    Este punto de vista muestra cómo las aplicaciones apoyan procesos de negocio o son utilizadas por otras aplicaciones. Puede emplearse tanto para diseñar aplicaciones como para identificar servicios necesarios dentro de un proceso operativo. \cite{definicion_capas}
}{imgs/Caso 4.pdf}{
    El caso representa un flujo de venta donde la transacción principal se ve influenciada por elementos previos como descuentos y promociones, y por opciones posteriores como pagos en línea o con tarjeta. El modelo refleja cómo interactúan las aplicaciones con el proceso, permitiendo identificar mejoras y dependencias relevantes.
}{0.8}{0.5}
